\documentclass[12pt]{article}
\usepackage{kotex}
\usepackage[utf8]{inputenc}
\usepackage{geometry}
\geometry{
	top = 20mm,
	bottom = 20mm,
	left = 20mm,
	right = 20mm
}
\setcounter{tocdepth}{1}
\title{\textbf{2019 가을학기 스터디 계획}}
\author{Sungchan Yi}

% \pagenumbering{gobble}
\renewcommand{\baselinestretch}{1.2}
\begin{document}
\maketitle

\tableofcontents
\subsection{공통 사항}
\begin{itemize}
	\item 참여 의사가 있으시면 저에게 어떤 방법으로든 의사를 밝혀 주시면 됩니다.
	\item 기본적으로 \textbf{온라인}에서 \textbf{각자 공부하는} 스터디 입니다. 오프라인에서 모임이 진행될 수도 있으나 필수는 아닙니다.
	\item 서로 도움을 주고 받고, 약간의 강제성을 부여하자는 취지입니다.
	\item 시작 날짜는 \textbf{9월 2일} 입니다.
	\item 재학생을 위해 시험기간에는 유동적으로 진행합니다.
	\item 슬랙(Slack)에서 진행됩니다. (\texttt{zxcvber.slack.com} 일 예정)
	\item 세부 규칙은 추후 변경될 수 있습니다.
	
\end{itemize}
\pagebreak
\section{알고리즘 스터디}
기초적인 알고리즘을 넘어서 중$\cdot$고급 알고리즘에 대해 학습해 봅니다.
\subsection{학습 내용}
\begin{itemize}
	\item Convex Hull
	\item Network Flow
	\item Segment Trees / Fenwick Trees
	\item String Matching
	\item Satisfiability Problem
	\item Suffix Arrays
	\item Tries
	\item Fast Fourier Transform
	\item Chinese Remainder Theorem
	\item ... and More!
\end{itemize}
\subsection{진행 방법 및 계획}
\begin{itemize}
	\item 아래 내용은 매주 진행됩니다.
	\item 적당한 토픽을 골라 그 토픽과 관련된 문제를 2 문제 풉니다.
	\item \texttt{solved.ac} 에서 계산된 자신의 티어를 고려하여 적당한 티어의 문제를 10 문제 이상 풉니다.
	\item 푼 문제 번호를 정리하여 슬랙에 보고합니다.
\end{itemize}
\pagebreak

\section{기계학습/인공지능 스터디}
기초적인 기계학습 이론에 대해 배우고 딥러닝을 해봅시다.
\subsection{강의}
\begin{itemize}
	\item 강의: Stanford Univ. \textbf{CS231n}: Convolutional Neural Networks for Visual Recognition
	\item Link: \texttt{http://cs231n.stanford.edu/} (2017년도 강의 유튜브로 시청 가능)
	\item Reference: Stanford Univ. \textbf{CS229}: Machine Learning
	\item Link: \texttt{http://cs229.stanford.edu/}
\end{itemize}
\subsection{참고 도서}
\begin{itemize}
	\item \textbf{Deep Learning} (\texttt{https://www.deeplearningbook.org/})
	\item Hands-On Machine Learning with Scikit-Learn \& TensorFlow
	\item Deep Learning from Scratch
\end{itemize}

\subsection{Prerequisites}
다음은 CS231n 강의에서 요구하는 사항입니다.
\begin{itemize}
	\item Python (C/C++ is also OK)
	\item 미적분학 / 선형대수학
	\item 기초 확률론 및 통계학
\end{itemize}

\subsection{진행 방법 및 계획}
\begin{itemize}
	\item Deep Learning 책과 CS229 강의자료로 기본적인 이론 학습
	\item CS231n 강의 매주 1 $\sim$ 2개 시청
	\item CS231n 과제 직접 구현해보기 (주 목표)
	\item ... 추가 예정
\end{itemize}
\end{document}
