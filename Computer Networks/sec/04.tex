\section{Network Layer I}
\subsection{Overview of Network Layer}
\subsubsection{Network Layer}
\begin{itemize}
	\item Transport segment from sending host to receiving host
	\item Sending side encapsulates segments (from transport layer) into \textbf{datagrams}\footnote{`Message' or `packet' that is on the network layer level} and sends them
	\item Receiving side receives datagrams and delivers the segments to transport layer
	\item Network layer protocols exists in \textbf{every} host and router
	\item Router examines header fields in all IP datagrams passing through it - the destination is determined by \textbf{IP address}
\end{itemize}

\subsubsection{Two Key Functions of Network Layer}
\paragraph{Routing}
\begin{itemize}
	\item Determines the route taken by packets from source to destination
	\item Routing algorithms help create the forwarding table
\end{itemize}

\paragraph{Forwarding}
\begin{itemize}
	\item Moves packets from router's input to appropriate router output
	\item Packet delivery to the next node
\end{itemize}

\paragraph{Traditional vs SDN Network}
\begin{itemize}
	\item In traditional IP networks, routing and forwarding is done in the same system (for each router and switch)
	\item In \textit{software-defined network}(SDN), routing function and forwarding function are separated - The central server determines the optimal route, and the switches do the forwarding according to the route given by the central server 
\end{itemize}

\subsection{Inside of Router}
\begin{itemize}
	\item router input ports $\rightarrow$ high-speed switching fabric + routing processor $\rightarrow$ router output ports
\end{itemize}

\subsubsection{Input Port Functions}
\begin{itemize}
	\item physical layer $\rightarrow$ data link layer $\rightarrow$ decentralized switching
	\item \textbf{Physical layer} determines if the input is 0 or 1 (bit level)
	\item \textbf{Data link layer} - Explained later in Chapter 5. Checks errors, etc.
	\item IP datagram is created after these two layers
	\item IP datagrams may not be processed in real-time, because switching fabric may be busy - datagrams are stored in a buffer called \textbf{input queue}
	\item \textbf{Decentralized switching}
	\begin{itemize}
		\item Using header field values, lookup output port using forwarding table in input port memory
		\item The goal is to complete input port processing at `line speed'
		\item \textit{Queueing} is done if datagrams arrive faster than forwarding rate into switch fabric
	\end{itemize}
\end{itemize}

\paragraph{Input Port Queueing}
\begin{itemize}
	\item Fabric may be slower than input ports combined. Then queueing occurs at input queues
	\item Queueing delay and loss may occur due to input buffer overflow
	\item \textbf{Head-of-the-line blocking}: queued datagram at front of the queue may prevent other datagrams in queue from moving forward
\end{itemize}

\subsubsection{Switching Fabrics}
\begin{itemize}
	\item Transferring packet from input buffer to appropriate output buffer
	\item \textit{Switching rate}: rate at which packets can be transferred from inputs to outputs
	\item Switching rate is often measured as a multiple of input/output line rate
	\item $N$ inputs: switching rate of $N$ times line rate is desirable
	\item Three types of switching fabrics: memory, bus, interconnection network
\end{itemize}

\paragraph{Switching via Memory}
\begin{itemize}
	\item First generation routers
	\item Traditional computers with switching under direct control of CPU
	\item Packets are copied to system's (main) memory, and the IP in the header is read and sent to the destination
	\item Speed is limited by memory bandwidth (2 bus crossings per datagram)
\end{itemize}

\paragraph{Switching via Bus}
\begin{itemize}
	\item Datagram from input port memory to output port memory via a shared bus (direct copying - faster)
	\item Any input port can be sent to any output port (decided by CPU)
	\item But because a shared bus is used, \textit{bus contention} occurs: switching speed is limited by bus bandwidth
	\item When sending a datagram, an input port and an output port will be used exclusively (limitation of speed, because of a shared bus)
\end{itemize}

\paragraph{Switching via Interconnection Network}
\begin{itemize}
	\item Multiple datagrams can be sent at the same time, if the datagrams do not share ports - if the output port has to be shared, collision is inevitable
	\item A lot faster than memory/bus method
	\item Banyan networks, crossbar, other interconnection nets are used for multiprocessing
	\item Banyan networks use less switches compared to crossbars so it may be cheaper, but more susceptible to HOL blocking or collisions 
\end{itemize}

\subsubsection{Output Port Functions}
\begin{itemize}
	\item \textbf{Buffering} is required when datagrams arrive from fabric faster than the transmission rate
	\item \textbf{Scheduling} discipline chooses among queued datagrams for transmission
	\item Scheduling: choosing the next packet to send on link
	\item \textbf{FIFO}\footnote{First in first out, or sometimes FCFS, for first come first served}: send in order of arrival to queue
	\item Discard policy: if packets arrive to a full queue, which packet should be drop?
	\begin{itemize}
		\item \textit{Tail drop}: Drop the arriving packet
		\item \textit{Priority}: Drop the packet on priority basis
		\item \textit{Random}: Drop a random packet
	\end{itemize}
\end{itemize}

\paragraph{Priority Scheduling}
\begin{itemize}
	\item Sends highest priority queued packet
	\item Keep two queues, for high/low priority packets, respectively
	\item Multiple classes, with different priorities
	\item Priority class may depend on marking or other header info such as IP source/destination, port numbers etc.
	\item Packets in the low priority queue will be transmitted if the high priority queue is empty - causes problems
\end{itemize}

\paragraph{Round Robin (RR) Scheduling}
\begin{itemize}
	\item Keep multiple classes and queues
	\item Scan all queues round and around. For each cycle, send a \textit{single} packet if the queue is not empty
	\item Each queue is given a chance to send its packets so this method is a bit different compared to priority scheduling
	\item But if there are less (kinds of) packets corresponding to some class, the packets are likely to be processed a lot faster (less delay)
\end{itemize}

\paragraph{Weighted Fair Queueing (WFQ) Scheduling}
\begin{itemize}
	\item Generalized round robin: RR + weights
	\item Each class gets weighted amount of service in each cycle
\end{itemize}

\subsection{Internet Protocol Overview}
\begin{itemize}
	\item Network layer functions of host and router
	\item \textbf{Routing protocols} - path selection, RIP, OSPF, BGP\footnote{Routing information protocol, open shortest path first, border gateway protocol}
	\item \textbf{IP protocol} - addressing conventions, datagram format, packet handling conventions
	\item \textbf{ICMP\footnote{Internet control message protocol} protocol} - error reporting, router signaling
\end{itemize}

\subsubsection{IP Datagram Format}
\begin{itemize}
	\item 20 bytes of header + options (if necessary)
	\begin{itemize}
		\item \textit{Version}: 4 or 6
		\item \textit{Header length}: offset of data / length of the header + options
		\item \textit{Type of service}: multimedia data, real-time data... rarely used
		\item Datagram length: length of the datagram
		\item \textit{16-bit identifier, flags, fragment offset}: For fragmentation and reassembly of segments in routers
		\item \textit{Time to live}(TTL): maximum number of remaining hops, decremented at each router
		\item \textit{Upper layer}: Upper layer protocol to deliver payload to (TCP or UDP)
		\item \textit{Header checksum}: cycle redundancy check (CRC)
		\item \textit{Source/Destination IP address}: 32 bit
	\end{itemize}
\end{itemize}

\subsubsection{IP Fragmentation \& Reassembly}
\begin{itemize}
	\item Network links have \textit{maximum transfer unit} (MTU) 
	\item Each network uses different link types, so they have different MTUs
	\item Large IP datagrams should be divided (fragmented) within the network
	\item Reassembly will occur at the final destination
	\item IP header bits are used to identify and order related fragments
	\item \textit{Identifier}: Datagrams fragmented from large datagrams contain the same ID
	\item \textit{Fragment offset}: Offset of this datagram's data before fragmentation. 13 bits are used for offset, so the value is divided by 8 (flag uses 3 bits)\footnote{MTU - (header size) is divided by 8 to calculate the offset}
	\item \textit{Flag}: 3 bits. First bit is reserved, second bit is for \textit{don't fragment flag} (DF), last bit is for \textit{more fragment flag} (MF)
	\item If MF flag of a packet is 0, it is the last datagrams among fragmented datagrams.
\end{itemize}

\subsubsection{TTL}
\begin{itemize}
	\item TTL field (8 bytes)
	\item Included to ensure that datagrams do not circulate forever (ex. a long-lived routing loop) in the network
	\item Decremented by one each time the datagram is processed by a router
	\item If TTL reaches 0, a router must drop that datagram
\end{itemize}

\subsection{IP Addressing}
\begin{itemize}
	\item \textbf{IP address}: 32 bit identifier for host, router interface
	\item \textit{Interface}: connection between host/router and physical link
	\begin{itemize}
		\item Routers typically have multiple interfaces
		\item Hosts typically have one or two interfaces
	\end{itemize}
	\item IP address associated with each interface
	\item IP address is \textit{hierarchically addressed}
\end{itemize}

\subsubsection{IP Address Class}
\begin{itemize}
	\item \textbf{ICANN}(Internet Corporation for Assigned Names and Numbers): allocates addresses, manages DNS, assigns domain names, resolves disputes
	\item IP: \textit{network ID} + \textit{host ID}
	\begin{itemize}
		\item \textit{Network ID}: ID representing a network group, or a group of hosts
		\item \textit{Host ID}: ID of each host in a network group
	\end{itemize}
	\item Three classes: A, B, C
	\begin{itemize}
		\item Class A: starts with bit 0, network 7 bits + host 24 bits (First byte 0 \(\sim\) 127)
		\item Class B: starts with bit 10, network 14 bits + host 16 bits (First byte 128 \(\sim\) 191)
		\item Class C: starts with bit 110, network 21 bits + host 8 bits (First byte 192 \(\sim\) 223)
	\end{itemize}
\end{itemize}

\subsubsection{Subnets}
\begin{itemize}
	\item \textbf{Subnet}: a logical sub-division of an IP network
	\item Datagram forwarding is performed by routers
	\item Hosts in the same network can reach each other without intervening router
	\item Too many hosts in a network may increase maintenance overhead
	\item \textit{Divide and conquer}
	\item Division of IP address: Host ID = Subnet number + Host number
	\item Deciding the bit size of subnet number
	\begin{itemize}
		\item \textit{Subnet mask}: Indicates the bits that will be used as the network number
		\item If the bit at some position is 1, the bit will be used as the network number
		\item Ex. 255.255.255.0 \(\rightarrow\) 24 bits will be used as the network number
		\item The actual bit size of the subnet number will be: number of 1s in subnet mask - bit size of network ID (depending on IP address class)
	\end{itemize}
\end{itemize}

\subsubsection{Classless Inter-Domain Routing (CIDR)}
\begin{itemize}
	\item \textit{Supernetting}
	\item Limitations of IP address classes
	\begin{itemize}
		\item Class A: 128 possible network IDs, too small
		\item Class B: approximately 16,000 network IDs \(\rightarrow\) small nowadays
		\item Class C: Only 256 possible hosts, too small
	\end{itemize}
	\item CIDR address format: \texttt{A.B.C.D/x} where \texttt{x} is the number of bits in the network portion of address
	\item Multiple class C addresses can be grouped together to function as a single network group
	\item The hosts in each class C address can be assigned to individual hosts inside the network
\end{itemize}

\subsection{Datagram Forwarding}
\begin{itemize}
	\item IP packets are often called datagrams, hence IP network is often called datagram network
	\item IP network has no call setup (recall that TCP had handshakes)
	\item Router looks at the destination address in the datagram and decides the next route for the datagram
	\item Routers don't have any state about end-to-end connections
	\item 2 methods of datagram forwarding
	\item \textit{Destination based forwarding}: Based only on destination IP address (traditional)
	\begin{itemize}
		\item What is the next most appropriate node, considering the destination address?
	\end{itemize}
	\item \textit{Generalized forwarding}: Based on any set of header field values (SDN)
	\begin{itemize}
		\item Generalized, in the sense that factors other than the destination address can be taken into account when the datagram is forwarded
		\item For instance, load balancing, congestion etc. can be taken into account
		\item This is made possible by software defined networks
		\item More about this later
	\end{itemize}
\end{itemize}

\subsubsection{Destination Based Forwarding}
\begin{itemize}
	\item When a datagram arrives, the router will look at the destination address
	\item Then with the local forwarding table, the router will decide the output link
	\item But since there are \(2^{32}\) IP addresses, it is inefficient to store all of them inside the table
	\item Instead, it will only contain the \textit{network ID}
	\item When the datagram is forwarded to the destination network, the destination network will forward the datagram to the host
	\item Can we do better? There are \(2^{21}\) class C addresses...
	\item The router will send a range of destination addresses to a single output link
	\item \textbf{Longest prefix matching}: when looking for a forwarding table entry for given destination address, use the \textit{longest} address prefix that matches the destination address
	\begin{itemize}
		\item The longest matching prefix will have more information about the destination
		\item So the longest matching prefix will be the most appropriate output link
	\end{itemize}
\end{itemize}

\subsection{Dynamic Host Configuration Protocol}
\begin{itemize}
	\item Host dynamically obtains its IP address from the network server when it joins the network
	\begin{itemize}
		\item Allows reuse of addresses (only hold the address while connected)
		\item Can renew its lease on addresses in use
		\item Support for mobile users who want to join the network
		\item Dynamic allocation of IP addresses for each connected user
	\end{itemize}
\end{itemize}

\subsubsection{DHCP Overview}
\begin{enumerate}
	\item \textbf{DHCP Discover}: host broadcast
	\begin{itemize}
		\item Broadcast: Is there a DHCP server out there?
		\item Destination address: \texttt{255.255.255.255} (broadcast to all devices on network)
	\end{itemize}
	\item \textbf{DHCP Offer}: DHCP server responds
	\begin{itemize}
		\item Broadcast: I'm a DHCP server, here's an IP address you can use!
		\item \texttt{yiaddrr} contains the IP address to use
		\item The client checks transaction ID to realize that this is a response to its request\footnote{DHCP offer is also broadcast to all devices on network}
	\end{itemize}
	\item \textbf{DHCP Request}: Host requests IP address
	\begin{itemize}
		\item Broadcast: OK, I'll take that IP address!
	\end{itemize}
	\item \textbf{DHCP ACK}: DHCP server sends IP address
	\begin{itemize}
		\item Broadcast: OK, you can use that IP address!
	\end{itemize}
\end{enumerate}
\begin{itemize}
	\item IP obtained from DHCP has a \textit{lifetime}. Should be renewed before expiration
	\item Why take 4 steps? Can't we use the address given from the offer?
	\begin{itemize}
		\item There may be multiple DHCP servers on the network
		\item Client will receive multiple offers from multiple servers
		\item The client decides on which offer to take, and requests it to the server
		\item \textit{Broadcast} - the decision will be broadcast, to let other devices on the network know about this decision
	\end{itemize}
	\item DHCP server will give client's IP address, IP address of the first-hop router for client, and name/address of DNS server - as these information are necessary for the client to connect to the network
\end{itemize}

\subsubsection{Advantages of DHCP}
\begin{itemize}
	\item Efficient use of IP addresses
	\begin{itemize}
		\item IPv4 address exhaustion
		\item If every device used static IPv4 addresses, each device would have to hold on to that address even when it is not connected to the network
		\item IP is also considered as a network resource, should be used efficiently
	\end{itemize}
	\item Allows the ``plug \& play" of a computer system
	\begin{itemize}
		\item In the past, we had to manually type in static IP addresses, subnet masks, default gateway
		\item Dynamic IP, DNS server address allocation provides abstraction for people without network knowledge
		\item Plug in the network cable and everything is set up automatically
	\end{itemize}
\end{itemize}

\subsection{Network Address Translation}
\begin{itemize}
	\item In the past, each household used a single IP for its desktop computer. Nowadays, each household has more network devices such as smartphones etc. How to we assign IP addresses to all the devices?
	\item \textit{Motivation}: local network uses just one IP address as far as outside world is concerned...
	\begin{itemize}
		\item Range of IP addresses is not needed from ISP - just one IP address for all devices
		\item But this IP address gets translated inside the local network, without notifying the outside world
		\item Can change ISP without changing addresses of devices in local network
		\item Devices inside local network is not explicitly addressable or visible by the outside world (security)
	\end{itemize}
\end{itemize}

\subsubsection{Network Address Translation}
\begin{itemize}
	\item \textbf{Private IP}: Meaningful only in the local/home network
	\begin{itemize}
		\item A class: \texttt{10.x.x.x}
		\item B class: \texttt{172.16.x.x} \(\sim\) \texttt{172.31.x.x}
		\item C class: \texttt{192.168.0.x} \(\sim\) \texttt{192.168.255.x}
	\end{itemize}
	\item All datagrams leaving the local network have the same single source NAT IP address (\textbf{public IP}), but different source port numbers
	\item Datagrams with source or destination in the network use private IP for source and destination
\end{itemize}

\begin{enumerate}
	\item Host in local network sends datagram to destination, using \textit{private IP} as source
	\item The router uses \textbf{NAT translation table} to change datagram source address from private IP to \textit{public IP} (update table if necessary)
	\item Reply arrives with destination address using the \textit{public IP}
	\item The router uses NAT translation table again to change the destination address of datagram from public IP to \textit{private IP}
	\item The host in local network receives the datagram
\end{enumerate}

\subsection{IPv6}
\begin{itemize}
	\item Motivation
	\begin{itemize}
		\item DHCP, NAT partially solved the IPv4 address exhaustion issue, but it isn't the fundamental solution
		\item The number of addresses should be increased
	\end{itemize}
	\item Additional Motivation
	\begin{itemize}
		\item Network technology has advanced since the use of IPv4
		\item Why not change the header format to speed up processing and forwarding?
		\item Header changes to facilitate QoS
	\end{itemize}
\end{itemize}

\subsubsection{IPv6 Datagram Format}
\begin{itemize}
	\item Uses fixed-length 40 byte header
	\item Comparison with IPv4 datagram
	\begin{itemize}
		\item IPv4 used 20 bytes, with 8 bytes for source and destination. Other fields consist the remaining 12 bytes
		\item IPv6 address is 16 bytes, using 32 bytes for source and destination address - this is inevitable, and the remaining 8 bytes are used to contain other values, which is a \textit{decrease}
	\end{itemize}
	\item No fragmentation allowed (fragmentation offset, flag, id is gone)
	\item \textit{version}: either v4 or v6
	\item \textit{priority}: identify priority among datagrams in flow
	\begin{itemize}
		\item Router does priority scheduling
		\item On buffer overflow, throw away packets with lower priority
	\end{itemize}
	\item \textit{flow label}: identify datagrams in same `flow'
	\begin{itemize}
		\item IPv4 used to consider each datagrams independently
		\item Define a flow label between source/destination processes, and the datagrams between these processes will have the same flow label
		\item Treats subsequent IP datagrams as a whole
		\item Important in software defined networks (SDN)
	\end{itemize}
	\item \textit{next header}: identify upper layer protocol for data (UDP/TCP)
	\item \textit{hop limit}: same as TTL in IPv4
	\item Checksum has been removed entirely to reduce processing time at each hop
	\begin{itemize}
		\item We use optical fibers now - reduced error rate
		\item TTL value decreases every hop, thus checksum has to be recalculated every hop
		\item Checking and recalculating checksum is an overhead
	\end{itemize}
\end{itemize}

\subsubsection{Migration from IPv4 to IPv6}
\begin{itemize}
	\item Not all routers can be upgraded simultaneously
	\item How to operate with mixed IP versions?
	\item \textbf{Tunneling}: IPv6 datagram is carried as payload in IPv4 datagram among IPv4 routers
	\begin{itemize}
		\item Pad IPv4 header onto the IPv6 datagram
		\item The IPv6 routers should provide dual protocol
	\end{itemize}
	\item Long time for deployment!
	\item Necessary for the future - preparation for the IoT era, more objects will need IP addresses
\end{itemize}

\pagebreak