\section{Network Layer I}
\subsection{Overview of Network Layer}
\subsubsection{Network Layer}
\begin{itemize}
	\item Transport segment from sending host to receiving host
	\item Sending side encapsulates segments (from transport layer) into \textbf{datagrams}\footnote{`Message' or `packet' that is on the network layer level} and sends them
	\item Receiving side receives datagrams and delivers the segments to transport layer
	\item Network layer protocols exists in \textbf{every} host and router
	\item Router examines header fields in all IP datagrams passing through it - the destination is determined by \textbf{IP address}
\end{itemize}

\subsubsection{Two Key Functions of Network Layer}
\paragraph{Routing}
\begin{itemize}
	\item Determines the route taken by packets from source to destination
	\item Routing algorithms help create the forwarding table
\end{itemize}

\paragraph{Forwarding}
\begin{itemize}
	\item Moves packets from router's input to appropriate router output
	\item Packet delivery to the next node
\end{itemize}

\paragraph{Traditional vs SDN Network}
\begin{itemize}
	\item In traditional IP networks, routing and forwarding is done in the same system (for each router and switch)
	\item In \textit{software-defined network}(SDN), routing function and forwarding function are separated - The central server determines the optimal route, and the switches do the forwarding according to the route given by the central server 
\end{itemize}

\newpage
\subsection{Inside of Router}

\subsection{Internet Protocol Overview}

\subsection{IP Addressing}

\subsection{Datagram Forwarding}

\subsection{Dynamic Host Configuration Protocol}

\subsection{Network Address Translation}

\subsection{IPv6}
