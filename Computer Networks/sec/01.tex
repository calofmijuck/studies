\section{Introduction}
\subsection{What is the Internet?}

\subsection{Network Edge}

\subsection{Network Core}

\subsection{Internet Structure}

\subsection{Performance Metrics}
\subsubsection{Evaluation Metrics}
\begin{itemize}
	\item \textbf{Delay}: Packet delivering time from source to destination
	\item \textbf{Packet loss}: Ratio of lost packets to total sent packets\footnote{PDR: Packet delivery ratio}
	\begin{itemize}
		\item If the queue is full, the arriving packets will be dropped
		\item Lost packets may be re-transmitted by previous nodes, by source end system, or not at all
	\end{itemize}
	\item \textbf{Throughput}: Amount of traffic delivered / unit time
	\begin{itemize}
		\item Rate at which bits are transferred between source and destination
		\item Can be measured instantaneously, or on average
		\item \textit{Bottleneck link}: The link on end-end path that constrains the throughput (usually the one with minimum capacity)
	\end{itemize}
\end{itemize}

\subsubsection{Four sources of delay}
$$d_{nodal} = d_{proc} + d_{queue} + d_{trans} + d_{prop}$$
\begin{itemize}
	\item Queueing Delay
	\begin{itemize}
		\item Time waiting at output buffer for transmission
		\item \textbf{Congestion} dependent
	\end{itemize}
	\item Transmission Delay
	\begin{itemize}
		\item $d_{trans} = L / R$ where $L$: packet length (bits), $R$: link bandwidth (bps)
	\end{itemize}
	\item Processing Delay
	\begin{itemize}
		\item Bit error checking
		\item Decision of output link
		\item Typically takes less than a few milliseconds (hardware acceleration)
	\end{itemize}
	\item Propagation Delay
	\begin{itemize}
		\item $d_{prop} = d / s$ where $d$: length of physical link, $s$: signal speed
	\end{itemize}
\end{itemize}
Queueing and transmission delay take up most of the delay.

\subsubsection{More on Queueing Delay}
$$\textsf{(Traffic Intensity)} = \frac{La}{R}$$
where $R$ is the link bandwidth (bps) or transmission rate, $L$ is the average packet length (bits), $a$ is the average packet arrival rate. As the traffic intensity $\rightarrow$ 1, the average queueing delay will grow without bound.

\subsection{Protocol Stack}
A communication protocol stack is composed of several \textbf{layers}. Each layer implements a service via its own internal actions, and by relying on services provided by the underlying layers.\\
Layering or \textbf{modularization} eases development, maintenance, and updating of the whole system. But this can be harmful in cases when a higher level layer needs information from the lower layers.\footnote{Consider a navigation system, which uses ``physical'' information like actual traffic, when choosing the fastest route between two places. But this ``physical'' information wouldn't normally be visible to other layers.}

\subsubsection{Internet Protocol Stack}
\begin{enumerate}
	\item \textbf{Physical}: Bits on the wire
	\item \textbf{Link}: Data transfer between neighboring network elements
	\item \textbf{Network}: Routing of datagrams from source to destination
	\item \textbf{Transport}: Process to process data transfer
	\item \textbf{Session}: Synchronization, connection management, recovery of data exchange
	\item \textbf{Presentation}: Allows applications to interpret the meaning of data
	\item \textbf{Application}: Supporting network applications
\end{enumerate}

\subsection{Network Security}
The field of network security arises from these questions:
\begin{itemize}
	\item How can bad guys attack our computer networks?
	\item How do we defend our networks against those attacks?
	\item How do we design architectures that are immune to attacks?
\end{itemize}

\subsubsection{Forms of Attacks}
\begin{itemize}
	\item Malware
	\begin{itemize}
		\item virus: A self-replicating infection by receiving or executing an object
		\item worm: A self-replicating infection by passively receiving object that gets itself executed
		\item spyware
		\item ransomware
	\end{itemize}
	\item Packet Sniffing: Promiscuous network interface reads/records all packets passing by
	\item Denial of Service: Attackers make resources unavailable by sending a huge amount of fake traffic
	\item Fake Addresses: Send packets with fake source address
	\item Fake Wi-Fi AP: Steal user's credentials using fake AP
\end{itemize}

\subsection{History of the Internet}
\begin{itemize}
	\item Firstly developed as ARPAnet
	\item Internetworking architecture = autonomy + minimalism
	\item TCP/IP, WWW
\end{itemize}