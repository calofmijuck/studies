\section{Introduction}
\subsection{What is the Internet?}
\begin{itemize}
	\item \textbf{Internet} = Inter- + net(work)
	\item Network of networks
	\item Various types of networks: mobile network, home network, global ISP, regional ISP
\end{itemize}

\subsubsection{Components}
\begin{itemize}
	\item HW components
	\begin{itemize}
		\item End hosts
		\item Links: copper, fiber, radio, satellite
		\item Interconnection devices: router, switch, repeater
	\end{itemize}
	\item SW components
	\begin{itemize}
		\item Operating software
		\item Application programs
		\item \textbf{Protocols}
	\end{itemize}
\end{itemize}

\subsubsection{Protocols}
\begin{itemize}
	\item A defined set of rules and regulations that determine how data is transmitted in telecommunications and computer networking
	\item All communication activity in Internet is governed by protocols
	\item Protocols define
	\begin{itemize}
		\item Message format
		\item Order of messages sent and received among network entities
		\item Actions taken on message transmission and receipt
	\end{itemize}
\end{itemize}

\subsection{Network Edge}
Network edge contain hosts, which are clients and servers.

\subsection{Network Core}
Network core is a mesh of interconnected routers or switches. They forward packets from one router (or switch) to the next along the path from the source to destination

\subsubsection{Switching Mechanisms}
\begin{itemize}
	\item \textbf{Circuit Switching}
	\begin{itemize}
		\item End to end resources reserved for communication between source and destination
		\item Entire data flow through along the path like water
		\item Resources are dedicated to each connection - the circuit segment is idle if it is note being used
		\item Common in traditional telephone networks
		\item Channel allocation methods: frequency division multiplexing (FDM), time division multiplexing (TDM)
	\end{itemize}
	\item \textbf{Packet Switching}
	\begin{itemize}
		\item Entire data is broken into small packets
		\item Each packet has its destination address
		\item Each packet is handled independently
		\item Packet is transmitted at full link capacity
		\item Takes $L/R$ seconds to transmit $L$-bit packet into link at $R$ bps
		\item \textit{Store and forward}: entire packet must arrive at the router before it can be forwarded to the next
		\item End to end delay $\approx 2L/R$
		\item If arrival rate of packets exceed the transmission rate of link, packets can be dropped (lost) if the packet queue inside the router is full
	\end{itemize}
	\item Comparison
	\begin{itemize}
		\item Packet switching allows more users to use the network
		\item Circuit switching guarantees the quality of service for each connection
	\end{itemize}
\end{itemize}

\subsection{Internet Structure}
Nobody or everybody is in charge of the Internet.
\begin{itemize}
	\item End systems connect to the Internet via \textbf{access ISPs} (Internet Service Providers)
	\item Access ISPs in turn must be interconnected so that any two hosts can communicate with each other
	\item Resulting network of networks is very complex
\end{itemize}

\subsubsection{Connecting Access ISPs}
\begin{itemize}
	\item How do we \textit{interconnect} the access ISPs?
	\item Connecting each access ISP to every other one would not be scalable, since we need $\mathcal{O}(n^2)$ connections
	\item Better: Keep a \textit{global transit ISP} and connect each access ISP to it
	\item Competing global transit ISPs appear and the are also interconnected by peering link and Internet exchange point (IXP)
	\item Regional networks arise to connect access networks to global ISPs
\end{itemize}



\subsection{Performance Metrics}
\subsubsection{Evaluation Metrics}
\begin{itemize}
	\item \textbf{Delay}: Packet delivering time from source to destination
	\item \textbf{Packet loss}: Ratio of lost packets to total sent packets\footnote{PDR: Packet delivery ratio}
	\begin{itemize}
		\item If the queue is full, the arriving packets will be dropped
		\item Lost packets may be re-transmitted by previous nodes, by source end system, or not at all
	\end{itemize}
	\item \textbf{Throughput}: Amount of traffic delivered / unit time
	\begin{itemize}
		\item Rate at which bits are transferred between source and destination
		\item Can be measured instantaneously, or on average
		\item \textit{Bottleneck link}: The link on end-end path that constrains the throughput (usually the one with minimum capacity)
	\end{itemize}
\end{itemize}

\subsubsection{Four sources of delay}
$$d_{nodal} = d_{proc} + d_{queue} + d_{trans} + d_{prop}$$
\begin{itemize}
	\item Queueing Delay
	\begin{itemize}
		\item Time waiting at output buffer for transmission
		\item \textbf{Congestion} dependent
	\end{itemize}
	\item Transmission Delay
	\begin{itemize}
		\item $d_{trans} = L / R$ where $L$: packet length (bits), $R$: link bandwidth (bps)
	\end{itemize}
	\item Processing Delay
	\begin{itemize}
		\item Bit error checking
		\item Decision of output link
		\item Typically takes less than a few milliseconds (hardware acceleration)
	\end{itemize}
	\item Propagation Delay
	\begin{itemize}
		\item $d_{prop} = d / s$ where $d$: length of physical link, $s$: signal speed
	\end{itemize}
\end{itemize}
Queueing and transmission delay take up most of the delay.

\subsubsection{More on Queueing Delay}
$$\textsf{(Traffic Intensity)} = \frac{La}{R}$$
where $R$ is the link bandwidth (bps) or transmission rate, $L$ is the average packet length (bits), $a$ is the average packet arrival rate. As the traffic intensity $\rightarrow$ 1, the average queueing delay will grow without bound.

\subsection{Protocol Stack}
A communication protocol stack is composed of several \textbf{layers}. Each layer implements a service via its own internal actions, and by relying on services provided by the underlying layers.\\
Layering or \textbf{modularization} eases development, maintenance, and updating of the whole system. But this can be harmful in cases when a higher level layer needs information from the lower layers.\footnote{Consider a navigation system, which uses ``physical'' information like actual traffic, when choosing the fastest route between two places. But this ``physical'' information wouldn't normally be visible to other layers.}

\subsubsection{Internet Protocol Stack}
\begin{enumerate}
	\item \textbf{Physical}: Bits on the wire
	\item \textbf{Link}: Data transfer between neighboring network elements
	\item \textbf{Network}: Routing of datagrams from source to destination
	\item \textbf{Transport}: Process to process data transfer
	\item \textbf{Session}: Synchronization, connection management, recovery of data exchange
	\item \textbf{Presentation}: Allows applications to interpret the meaning of data
	\item \textbf{Application}: Supporting network applications
\end{enumerate}

\subsection{Network Security}
The field of network security arises from these questions:
\begin{itemize}
	\item How can bad guys attack our computer networks?
	\item How do we defend our networks against those attacks?
	\item How do we design architectures that are immune to attacks?
\end{itemize}

\subsubsection{Forms of Attacks}
\begin{itemize}
	\item Malware
	\begin{itemize}
		\item virus: A self-replicating infection by receiving or executing an object
		\item worm: A self-replicating infection by passively receiving object that gets itself executed
		\item spyware
		\item ransomware
	\end{itemize}
	\item Packet Sniffing: Promiscuous network interface reads/records all packets passing by
	\item Denial of Service: Attackers make resources unavailable by sending a huge amount of fake traffic
	\item Fake Addresses: Send packets with fake source address
	\item Fake Wi-Fi AP: Steal user's credentials using fake AP
\end{itemize}

\subsection{History of the Internet}
\begin{itemize}
	\item Firstly developed as ARPAnet
	\item Internetworking architecture = autonomy + minimalism
	\item TCP/IP, WWW
\end{itemize}

\pagebreak