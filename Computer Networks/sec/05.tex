\section{Network Layer II}
\subsection{Introduction to Routing}
\begin{itemize}
	\item \textbf{Routing}: determine the route taken by packets from source to destination
	\item \textbf{Forwarding}: move packets from router's input to appropriate router output
	\item \textbf{Control plane}: routing algorithm determines end to end path through network
	\item Data plane: forwarding table determines local forwarding at this router
	\item \textit{Per-router control plane}: individual routing algorithm components in each and every router interact with each other in control plane to compute forwarding tables
	\item \textit{Logically centralized control plane}: a distinct (typically remote) controller interacts with local control agents (CA) in routers to compute forwarding tables
	\begin{itemize}
		\item A remote controller controls the forwarding process
		\item Forwarding path decided with more information, thus efficient
		\item Often used in software defined networks
	\end{itemize}
\end{itemize}

\subsubsection{Routing Protocols}
\begin{itemize}
	\item \textbf{Goal of Routing}: determine `good' paths from source to destination through network of routers
	\begin{itemize}
		\item \textit{Path}: a sequence of routers that packets will traverse, in going from initial source to final destination
		\item Paths that take the least cost, that are fastest, that are the least congested
	\end{itemize}
	\item Abstraction as a graph \(G = (N, E)\) (can be a directed graph)
	\begin{itemize}
		\item \(N\) is the set of vertices - \textit{routers} in this case
		\item \(E\) is the set of edges - \textit{links} in this case
		\item Cost function \(c: N\times N \rightarrow \mathbb{R}\) where \(c(v, w)\) is the cost of using the link between router \(v\) and \(w\)
	\end{itemize}
\end{itemize}

\subsubsection{Routing Algorithm Classification}
\paragraph{Static vs Dynamic}
\begin{itemize}
	\item Static
	\begin{itemize}
		\item Routes are changed manually by the administrator
	\end{itemize}
	\item Dynamic
	\begin{itemize}
		\item Routes are changed more quickly
		\item Periodic update in response to link cost changes
	\end{itemize}
\end{itemize}

\paragraph{Global vs Decentralized}
\begin{itemize}
	\item Global
	\begin{itemize}
		\item All routers have complete topology of the network and all link cost information
		\item \textit{Link state algorithms} 
	\end{itemize}
	\item Decentralized
	\begin{itemize}
		\item Router knows directly connected neighbors
		\item Router knows the path costs to other nodes, and the next router that should be taken to reach some other node
		\item \textit{Distance vector algorithm}
	\end{itemize}
\end{itemize}

\subsection{Link State Routing}
\begin{itemize}
	\item Need to know the complete topology of the network - \textbf{link state advertisement}(LSA) message
	\begin{itemize}
		\item Contains neighbor node information
		\item Contains link information to others (capacity, delay, etc.)
	\end{itemize}
\end{itemize}

\paragraph{Operation}
\begin{enumerate}
	\item A router constructs LSA
	\begin{itemize}
		\item When a neighbor changes, link goes up or down (activated/deactivated)
		\item Periodically
	\end{itemize}
	\item The LSA is broadcast to \textbf{all routers} in the network
	\item Based on LSA from all other nodes, each router constructs the complete topology
	\item Least cost paths between two nodes are computed by \textbf{Dijkstra's algorithm}
\end{enumerate}

\subsubsection{Dijkstra's Algorithm}
\begin{itemize}
	\item Single source shortest path algorithm
	\item Computes least cost paths from the source to all other nodes
	\item \(k\) iterations will give the least cost path to \(k\) destinations
\end{itemize}

\paragraph{Steps}
\begin{enumerate}
	\item Let \(S = \{u\}\) (source)
	\item For all \(v \in N(S)\) (neighbor nodes) set \(d(v) = c(u, v)\). If \(v\notin N(S)\), set \(d(v) = \infty\)
	\item \texttt{while} \(S \neq N\)
	\begin{enumerate}
		\item Find \(w\notin S\) such that \(d(w)\) is minimum, and \(S \leftarrow S\cup \{w\}\)
		\item For all \(x\in N(w)\) and \(x\notin S\), update \(d(x) = \min\{d(x), d(w) + c(w, x)\}\)
	\end{enumerate}
	\item \(d(x)\) contains the cost of shortest path from \(u\) to \(x\)
\end{enumerate}

\subsection{Distance Vector Routing}
\subsubsection{Bellman-Ford Equation}
\begin{itemize}
	\item Let \(d_x(y)\) be the cost of the least-cost path from \(x\) to \(y\)
	\item Then,
	\[
		d_x(y) = \min_{v\in N(x)}\{d_v(y) + c(x, v)\}
	\]
\end{itemize}

\subsubsection{Distance Vector Algorithm}
\begin{itemize}
	\item Node \(x\) initially
	\begin{itemize}
		\item Knows cost to each neighbor \(v\): \(c(x, v)\)
		\item Maintains its neighbors' distance vectors. For \(v\in N(x)\), node \(x\) maintains
		\[
			d_v = \{d_v(y) : y\in N\}
		\]
	\end{itemize}
	\item From time to time, each node sends its own distance vector estimate to neighbors
	\item When \(x\) receives new distance vector estimate from neighbor, it updates its own distance vector using Bellman-Ford equation
	\[
		d_x(y) \leftarrow \min_{v \in N(x)} \{d_v(y) + c(x, v)\} \text{ for each node } y\in N
	\]
	\item Each node
	\begin{itemize}
		\item \textit{Waits} for changes in local link cost or messages from neighbor
		\item \textit{Recompute}s estimates
		\item If distance vector to any destination has changed, \textit{notifies} neighbors
	\end{itemize}
\end{itemize}

\paragraph{Good News Travels Fast}
\begin{itemize}
	\item When link cost changes,
	\begin{itemize}
		\item Nodes detect local link cost change
		\item Nodes update routing info, and recalculates distance vector
		\item If distance vector changes, notify neighbors
	\end{itemize}
	\item Link cost decrease travels fast to other nodes
	\item Link cost increase travels slowly to other nodes - \textit{count to infinity} problem
	\begin{itemize}
		\item Takes a lot of iterations before algorithm stabilizes 
	\end{itemize}
\end{itemize}

\subsubsection{Link State vs Distance Vector}
\begin{center}
	\begin{tabular}{l|l}
		\textbf{Link State} & \textbf{Distance Vector} \\ \hline
		Incremental updates & Entire routing table is sent as an update \\
		Fast convergence & Slow convergence \\
		Updates are broadcast to entire network & Updates are sent to directly connected neighbor only \\
		Routers know the entire network topology of that area & Routers don't know the entire network topology\\
		No routing loops & Count to infinity problem\\
		Complex configuration (for large network) & Simple configuration \\
		Example: \textit{OSPF} (Open Shortest Path First) & Example: \textit{RIP} (Routing Information Protocol)
	\end{tabular}
\end{center}

\subsection{Intra-AS Routing: OSPF}
\subsubsection{Scalability of Routing}
\begin{itemize}
	\item So far, we have assumed that all routers are identical and the network is `flat'
	\item In practice, we have billions of destinations. We can't store all destinations in routing tables, and routing table exchange would swamp links
	\item There are \textbf{scalability issues} in routing
	\item Administrative autonomy
	\begin{itemize}
		\item Internet is a network of networks
		\item Each network admin may want to control routing in its own network
	\end{itemize}
\end{itemize}

\subsubsection{Autonomous System}
\begin{itemize}
	\item Aggregate routers into regions known as \textbf{autonomus system}s (AS)
	\item \textit{Autonomous system} is a collection of connected Internet protocol routing prefixes under the control of one or more network operators on behalf of a \textit{single administrative entity}
\end{itemize}

\paragraph{Intra-AS Routing}
\begin{itemize}
	\item Routing among hosts, routers in same AS (network)
	\item All routers in AS must run the same intra-domain protocol
	\item Routers in different AS can run different intra-domain routing protocol
	\item Gateway router (border router) at the edge of its own AS has links to routers in other AS
	\item RIP (distance vector), OSPF (link state)
\end{itemize}

\paragraph{Inter-AS Routing}
\begin{itemize}
	\item Routing among multiple AS
	\item Gateway routers perform inter-domain routing (as well as intra-domain routing)
	\item BGP (path vector)
\end{itemize}

\paragraph{Interconnected AS}
\begin{itemize}
	\item Forwarding table is configured by both intra-AS and inter-AS routing algorithm
	\begin{itemize}
		\item Intra-AS routing algorithm determines entries for destinations within the AS
		\item Inter-AS and intra-AS algorithm determine entries for external destinations
	\end{itemize}
\end{itemize}

\subsubsection{Routing Protocols}
\paragraph{Open Shortest Path First (OSPF)}
\begin{itemize}
	\item Most common intra-AS routing protocol
	\item `open': publicly available open protocol (documentation available)
	\item Uses link state algorithm
	\begin{itemize}
		\item LSAs to all other routers in entire AS - carried directly over IP
		\item Topology map is known at each node
		\item Route computation using Dijkstra's algorithm
	\end{itemize}
	\item Security issue: all OSPF messages should be authenticated, to prevent malicious intrusion
	\begin{itemize}
		\item OSPF packet contains authentication type and authentication
		\item If authentication fails, the message is ignored
		\item \textit{Black hole attack}: sends false information about shortest paths, which may cause congestion
	\end{itemize}
\end{itemize}

\paragraph{Hierarchical OSPF}
\begin{itemize}
	\item Hierarchical OSPF in large domains
	\item Minimizes routing update traffic
	\item Localizes the impact of topology changes to an area
	\item \textit{Two level hierarchy}: local area, backbone
	\begin{itemize}
		\item LSAs are only shared in the same area
		\item Each node has detailed area topology, only knows direction (shortest path) to nets in other areas
	\end{itemize}
	\item Area border routers: summarize distances to nets in own area, advertise to other area border routers
	\item Backbone routers: run OSPF routing limited to backbone area, and connect to other AS
\end{itemize}

\subsection{Inter-AS Routing: BGP}

\newpage


\subsection{Software Defined Networking}

\subsection{OpenFlow}

\subsection{Internet Control Message Protocol}

