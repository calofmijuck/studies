\section{Transport Layer}
\subsection{Transport Layer Services}
Some terminology:
\begin{itemize}
	\item \textbf{Program}: An executable file containing a set of instructions written to perform a specific job (usually stored on a disk)
	\item \textbf{Process}: An executing \textit{instance} of a program that resides on the primary memory. Several processes can be related to the same program at the same time
	\item \textbf{Thread}\footnote{\textbf{Thread} of execution is the smallest sequence of programmed instructions that can be managed independently by a scheduler, which is typically a part of the operating system.}: The smallest executable unit of a process
\end{itemize}

\subsubsection{Transport Layer Function}
\begin{itemize}
	\item Provides \textbf{logical communication between processes}
	\item The layer relies on and enhances services from the network layer\footnote{The network layer provides logical communication between \textbf{hosts}}
	\item Sending Side
	\begin{itemize}
		\item Applies \textbf{fragmentation} to application messages
		\item Passes segments\footnote{We see terms like \textit{frame}, \textit{datagram}, \textit{packet}, \textit{segment} when we study computer networks. They are similar, but we use layer-specific terms to represent the information unit in that layer. In the transport layer, we use the term \textbf{segment}.} to network layer
	\end{itemize}
	\item Receiving Side
	\begin{itemize}
		\item \textbf{Reassembles} segments into messages
		\item Passes the assembled message to the application layer
	\end{itemize}
\end{itemize}

\subsubsection{TCP and UDP}
\begin{itemize}
	\item \textbf{Transmission Control Procotol} (TCP)\footnote{Refer to 2.1.6.}
	\begin{itemize}
		\item \textbf{Reliable}, \textbf{in-order}\footnote{This doesn't mean that the order of sent messages is always preserved when receiving. (physically) In the application layer's point of view, it just \textit{seems} like it's received in the same order.} delivery
		\item \textbf{Connection oriented} service: connection setup, error control, flow control, congestion control
	\end{itemize}
	\item \textbf{User Datagram Protocol} (UDP)
	\begin{itemize}
		\item Unreliable, unordered delivery
		\item Connection-less service: faster than TCP
	\end{itemize}
\end{itemize}

\subsection{Multiplexing and Socket}
\subsubsection{Multiplexing and Demultiplexing}
\begin{itemize}
	\item This is the most basic role of the transport layer
	\item \textbf{Multiplexing} at sender: the sender sends data from its own multiple applications through network
	\item Data from multiple services are sent through a single shared channel
	\item \textbf{Demultiplexing} at receiver: the receiver delivers data packets to their appropriate receivers among its own multiple applications
	\item \textbf{Port numbers} are used for demultiplexing
	\begin{itemize}
		\item Different applications are assigned to different port numbers
		\item Transport layer segments have fields for source/destination port numbers in common
		\item Used to differentiate segments
	\end{itemize}
	\item Note that for connection oriented protocols (like TCP), the source IP and port are also used to differentiate each connection\footnote{Multiple applications can listen on the same port.}
	\item But how does the sender know the destination port on the receiver?
\end{itemize}

\subsubsection{Sockets}
\begin{itemize}
	\item API between application layer and transport layer
	\item Processes send/receive messages to/from its \textbf{socket}
	\item Analogous to door
	\begin{itemize}
		\item Sender passes message through the door
		\item Sender relies on transport infrastructure on other side of the door to deliver message to socket at the receiving process
	\end{itemize}
	\item The \textit{socket} is provided as the form of APIs by the operating system
\end{itemize}

\subsection{User Datagram Protocol}
\begin{itemize}
	\item Only provides the basic functions (multiplexing)
	\item \textbf{Connection-less} service
	\begin{itemize}
		\item Each UDP segment is handled independently of others
		\item \textit{Unreliable}: UDP segments may be lost or delivered out of order to app
	\end{itemize}
	\item But since UDP is fast, it is used for streaming multimedia applications, DNS, and SNMP
	\item For reliable transfer over UDP, the application must add that function (such as application specific error recovery)
	\item \textbf{UDP segment header}: Source port (16 bits), Destination port (16 bits), length, checksum, payload
	\item \textbf{Advantages}
	\begin{itemize}
		\item No connection establishment (no delay)
		\item Simple: No connection state at sender/receiver
		\item Small header size\footnote{Compare this to TCP headers.}
		\item No congestion control: UDP can blast away as fast as desired
	\end{itemize}
	\item \textbf{UDP Checksum}
	\begin{itemize}
		\item Detects transmission errors
		\item UDP doesn't have to provide reliable connections, but this checksum can be used to provide additional features elsewhere
		\item Sender creates a 16 bit integer checksum code of segment contents including the header
		\item The receiver will compute the checksum of the received message, and checks if the computed value is equal to the received value
	\end{itemize}
\end{itemize}

\subsubsection{Checksum Method}
\begin{itemize}
	\item Checksum is the 16-bit one's complement of the one's complement sum of a pseudo header of information from the IP header, the UDP header, and the data, padded with zero octets at the end (if necessary) to make a multiple of two octets.\footnote{RFC 768}
\end{itemize}

\subsection{Reliable Data Transfer Principles}

\subsection{Transmission Control Protocol}

\subsection{Congestion Control}

\subsection{TCP Congestion Control Algorithm}

\subsection{TCP vs UDP}