\section{Application Layer}
Previously, we have seen that there are 5 layers in the Internet protocol. We will go through each layer, in a top-down approach. We will cover the application layer in this section. 
\subsection{Principles of Application}
\subsubsection{Network Applications}
\begin{itemize}
	\item Types: email, web (server software, browser), P2P file sharing, SNS, messenger program, online games, streaming stored video (YouTube, Netflix)
	\item Run on (different) \textbf{end systems}: network core devices \textit{do not} run user applications. Ex) Routers only transfer information
	\item Communication over network (between end systems)
\end{itemize}

\subsubsection{Application Architectures}
There are two kinds of application structures: \textit{client-server} model, and \textit{peer-to-peer} model
\begin{itemize}
	\item \textbf{Client-Server Model}
	\begin{itemize}
		\item The \textbf{server} is \textit{always on}, has permanent IP address, since clients must be able to access the server anytime
		\item For scaling (to support a huge number of clients), a data center is typically built
		\item The \textbf{client} is a program that communicates with the server. It may be intermittently connected, and may have dynamic IP addresses. For communication, the client must send a request to the server first, using its IP address. Then the server will respond to that IP address.
		\item Clients do not communicate directly with each other
	\end{itemize}
	\item \textbf{Peer-to-Peer Model} (P2P)
	\begin{itemize}
		\item There is no \textit{always on} server. Each clients can function both as a client and a server. Arbitrary end systems will directly communicate with each other.
		\item Peers are intermittently connected and they can change IP addresses, so it is complex to manage.
		\item \textbf{Self scalability}: New peers bring new service capacity, as well as new service demands 
	\end{itemize}
\end{itemize}

\subsubsection{Application Layer Protocol}
The application layer is on the top of the Internet protocol. Then, the applications on this layer will communicate with the application layer on another end device. The protocol defined for this communication is called the \textbf{application layer protocol}. There are two types of messages that network application protocols exchange - \textbf{request}s and \textbf{response}s.
\begin{itemize}
	\item Message \textbf{Syntax}: What kinds of fields are there in the messages? How are the fields delineated?
	\item Message \textbf{Semantics}: What is the meaning of the information in the fields?
	\item Message \textbf{Pragmatics}: When and how do we process (send/respond) the messages?
\end{itemize}

\subsubsection{Application Protocol}
\begin{itemize}
	\item Open Protocols
	\begin{itemize}
		\item Standardized, open to public, for interoperability.\footnote{\textbf{Interoperability} is a characteristic of a product or system, whose interfaces are completely understood, to work with other products or systems, at present or in the future, in either implementation or access, without any restrictions.}
		\item Even if some non-authorized clients create a message that obeys the protocol, they can communicate with the server.
		\item Ex) HTTP, SMTP
	\end{itemize}
	\item Proprietary Protocols
	\begin{itemize}
		\item A protocol specific for some program
		\item Ex) Skype
		\item We cannot communicate with Skype without using the Skype application
	\end{itemize}
\end{itemize}

\subsubsection{Requirements of Network Applications}
\begin{center}
\begin{tabular}{c|c|c|c}
	\textbf{Application} & \textbf{Data Loss} & \textbf{Throughput} & \textbf{Time Sensitive} \\ \hline
	File Transfer & $\times$ & elastic & $\times$\\
	Email & $\times$ & elastic & $\times$\\
	Web Documents & $\times$ & elastic & $\times$\\
	Realtime Audio/Video & loss-tolerant & Audio: 5kbps$\sim$1Mbps / Video: 10kbps$\sim$5Mbps& 100ms\\
	Stored Audio/Video & loss-tolerant & (same) & A few secs\\
	Interactive Games & loss-tolerant & $\geq$ Few kbps & 100ms\\
	Text Messaging & $\times$ & elastic & Yes and no\\\hline
\end{tabular}
\end{center}~\\
The applications that require correct communication of information have \textit{elastic} throughput, since the correctness of the information is a lot more important than the speed of communication.\\
\\
These requirements should be met by the \textit{transport layer protocols}. The throughput and other requirements highly depend on the physical devices (routers, cables etc.) that hold up the network structure. The developers on the application layer cannot handle these requirements properly.

\subsubsection{Transport Protocol Services}
\begin{itemize}
	\item \textbf{TCP Service} (Transmission Control Protocol)
	\begin{itemize}
		\item \textbf{Error control}: In charge of reliable transport between sending and receiving processes
		\item \textbf{Flow control}: The sender won't overwhelm the receiver (not too much data)
		\item \textbf{Congestion control}: Throttle sender when network is overloaded
		\item \textit{Does not provide}: Timing, minimum throughput guarantee, security
		\item \textit{Connection oriented}: Setup is required between client and server processes
	\end{itemize}
	\item \textbf{UDP Service} (User Datagram Protocol)
	\begin{itemize}
		\item Unreliable data transfer between sending and receiving processes
		\item Why do we use UDP? - Some programs may require UDP. For example, the error controlling in TCP lowers the throughput, but UDP will ignore this error, resulting in faster communication, which is suitable for multimedia programs
	\end{itemize}
\end{itemize}

\subsection{Web and HTTP}
\textbf{Web pages} consist of \textbf{objects}. They can be an HTML file, JPEG image, Java applet, audio file, and more. Web page is described by \textbf{HTML-file}(s) which include several referenced objects. Each object is addressable by a \textbf{URL}(Uniform Resource Locator).
$$\overbrace{\texttt{www.someschool.edu}}^{\text{host name}}/\overbrace{\texttt{someDept/pic.gif}}^{\text{path name}}$$

\subsubsection{HTTP Overview}
\begin{itemize}
	\item \textbf{HTTP} (HyperText Transfer Protocol)
	\begin{itemize}
		\item Web's application layer protocol
		\item \textbf{Hyperlink}: A reference to data the reader can directly follow by clicking
		\item Uses client-server model
		\item Client uses a browser that requests, receives, and displays the Web objects
		\item The server is a web server that sends objects in response to request from clients
	\end{itemize}
	\item Based on \textbf{TCP}
	\begin{enumerate}
		\item Client initiates TCP connection (socket connection) to server
		\item Server accepts TCP connection from client
		\item HTTP messages are exchanged between brower and Web server
		\item TCP connection is closed
	\end{enumerate}
	\item \textit{HTTP reponse time} (\textbf{RTT}: Round trip time)
	\begin{itemize}
		\item 1 RTT to initiate TCP connection
		\item 1 RTT for HTTP request and first few bytes of HTTP response to return
		\item File transmission time
		\item Total = 2 RTT + file transmission time
	\end{itemize}
\end{itemize}

\subsubsection{HTTP Version}
\begin{itemize}
	\item \textbf{HTTP/1} (1996)
	\begin{itemize}
		\item \textit{Non-persistent} HTTP: Single object per single TCP connection
		\item Long latency
	\end{itemize}
	\item \textbf{HTTP/1.1} (1999)
	\begin{itemize}
		\item \textit{Persistent} HTTP: Multiple objects over on TCP connection
		\item Decreased latency
		\item \textit{Synchronous} order of response/request pairs over one TCP connection
	\end{itemize}
	\item \textbf{HTTP/2} (2015)
	\begin{itemize}
		\item \textit{Persistent} HTTP
		\item \textit{Asynchronous} (parallel) multiple response/request pairs over one TCP connection
	\end{itemize}
\end{itemize}

\subsubsection{HTTP Message}
There are two types of messages: \textbf{request}s and \textbf{response}s.
\begin{itemize}
	\item \textbf{Request Message}: Request line + Header lines + Body
	\item \textbf{Response Message}: Response line + Header lines + Body
	\item Request line: method, URL, version
	\item Response line: version, status code, status text
	\item Header lines: header field name, value
	\item Body: entity body
\end{itemize}

\subsubsection{REST}
\begin{itemize}
	\item \textbf{REpresentational State Transfer}
	\begin{itemize}
		\item HTTP should be \textbf{stateless}. If the state of client is kept in the server, it causes overhead for the server.
		\item The server will not store each client's states. Instead, the server will store the information about the client in the HTTP message. The server should also store the method to interpret the message. With all these information, the client knows how to fetch the data.
		\item We call a service \textbf{RESTful} if the service conforms to this architectural style
	\end{itemize}
	\item \textbf{Architectural Constraints}
	\begin{itemize}
		\item \textit{Client-server architecture}: Separation of the user interface concerns from the data storage concerns
		\item \textit{Statelessness}: No client context should be stored on the server between requests
		\item \textit{Cacheability}: Server responses are cachable on client and intermediaries
		\item \textit{Layered system}: One should be unable to tell whether a client is directly connected to the end server or to an intermediary along the way
		\item \textit{Uniform interface}: Simplification and decoupling of the architecture (Use of standardized languages - HTML, XML, JSON - that is not restricted by some computer architecture)
		\item \textit{Code on demand} (optional): Should be able to transfer executable code such as Java applets and JavaScript
	\end{itemize}
\end{itemize}

\subsection{Cookies and Web Caching}
\subsubsection{Cookies}
\textbf{Cookies} keep the states of clients.
\begin{enumerate}
	\item Client has a cookie file
	\item A usual HTTP request message is sent to the server (for the first time)
	\item The server creates an ID for the user, and creates an entry in the database
	\item The server responds with the ID, and tells the client to set the cookie with the given ID
	\item Now the client can send HTTP requests using that ID inside the cookie file
	\item The server (database) performs a cookie-specific action (distinguished by ID), and responds as usual
	\item A week later, (if the cookie still exists) the cookie can be used again for communication with the server
\end{enumerate}

\subsubsection{Web Caching}
\begin{itemize}
	\item For some servers (sites) with lots of visitors, request and responses for the exact same site would cause a huge overhead.
	\item The server prepares a \textbf{proxy server} that caches this information
	\begin{itemize}
		\item If the requested object is not in the cache, the proxy server requests the object from the origin server, and caches the data (and also responds to the client)
		\item Otherwise, the proxy server will used the cached data to respond to the client request
	\end{itemize}
	\item Effects of Web Caching
	\begin{itemize}
		\item For clients, the response time is reduced
		\item For servers, it can handle more users (reduced request overheads)
		\item For local ISPs, the traffic to external server is reduced, so the \textit{access link} can be used efficiently
	\end{itemize}
\end{itemize}

\subsection{SSL/TLS}
\subsubsection{Securing TCP}
\begin{itemize}
	\item We often use TCP and UDP for transport layer protocols. But when these protocols were developed back then, security considerations were not taken into account
	\item TCP and UDP have no encrpytion, even passwords traversed the Internet in clear text
	\item \textbf{SSL/TLS} provides encrypted TCP connection at the \textit{application layer}
	\item Assures data integrity\footnote{Data doesn't change during transmission}, and end-point authentication
	\item \textbf{SSL} (Secured Socket Layer)
	\begin{itemize}
		\item SSL v2.0 and v3.0 were released in 1995 and 1996, respectively
	\end{itemize}
	\item \textbf{TLS} (Transport Layer Security)
	\begin{itemize}
		\item Improved version of SSL v3.0
		\item More secure than SSL, but slower due to the two step communication processes
	\end{itemize}
\end{itemize}

\subsubsection{SSL/TLS Principle}
\begin{enumerate}
	\item When client connects to a server, the client asks for a secure SSL session
	\item The server sends a certificate\footnote{The client must check that this certificate is valid, and this certificate has to be signed by someone that the client trusts.}, that contains the server's public key
	\item The client sends a one time encryption key for the SSL session, encrypted with the server's public key
	\item The server decrypts the session key using its private key and establishes a secure session
	\item Now the client and the server can safely send/receive encrypted data
\end{enumerate}~
HTTPS is HTTP + SSL/TLS

\subsection{Electronic Mail}
\subsubsection{Electronic Mail}
\begin{itemize}
	\item There are 3 components that consist electronic mail
	\begin{itemize}
		\item \textit{User agents} (clients): edit mail, read mail
		\item \textit{Mail servers}: Google, Daum, Naver etc.
		\item \textit{Protocols}: SMPT, POP3, IMAP
	\end{itemize}
	\item Components of mail servers
	\begin{itemize}
		\item A mailbox for incoming messages
		\item A message queue for outgoing messages
		\item A protocol for exchanging mail between mail servers
	\end{itemize}
\end{itemize}

\subsubsection{SMTP Protocol}
\begin{itemize}
	\item \textbf{SMTP} (Simple Mail Transfer Protocol) [RFC 2821] uses TCP as the transport layer protocol for reliable email delivery
	\item Three phases of transfer
	\begin{itemize}
		\item Handshaking (Check sending server and receiving server)
		\item Transfer of messages
		\item Closure
	\end{itemize}
	\item Command/Response interaction (like HTTP)
	\begin{itemize}
		\item Commands are in ASCII text
		\item Reponses contain status codes and phrase
	\end{itemize}
\end{itemize}

\subsubsection{Mail Access Protocol}
\begin{itemize}
	\item \textbf{POP3} (Post Office Protocol 3)
	\begin{itemize}
		\item By default, deletes messages from the server after retrieving data
		\item Disconnects from the server after download 
		\item Needs to reconnect to the server on each download
	\end{itemize}
	\item \textbf{IMAP} (Internet Mail Access Protocol)
	\begin{itemize}
		\item Keeps all messages at the server and allows user to organize message folders
		\item Support synchronization across multiple devices
		\item Stays connected until the mail client app is closed and downloads messages on demand
	\end{itemize}
	\item \textbf{HTTP}
	\begin{itemize}
		\item Web-based email
		\item Used between browser and server
		\item Hotmail in the mid 1990s
		\item Google, Yahoo, etc.
	\end{itemize}
\end{itemize}

\subsection{Domain Name System}
To connect on the Internet, the client must know the \textit{address} of the server. This \textit{address} is called an \textbf{IP address}, and it is represented by 4 numbers between 0 and 225. But since these 4 numbers are hard to remember, (imagine having to memorize addresses for each website you use!) people use the \textit{name} of servers, instead of IP addresses.

\subsubsection{Domain Name System}
\begin{itemize}
	\item DNS \textbf{translates} an hostname to an IP address
	\item Procedure (Simplified)
	\begin{enumerate}
		\item The client asks the DNS for the IP address of some website
		\item The DNS responds with the IP address of that website
		\item The client uses that IP address to connect to the website
	\end{enumerate}
	\item Achieves \textit{load distribution} - many IP addresses correspond to one hostname (replicated Web servers), thus requests to the same hostname can be distributed between multiple servers
	\item \textbf{Distributed database system}
	\item De-centralized system
	\begin{itemize}
		\item Not scalable (for large number of clients)
		\item Single point of failure: The whole system breaks down when the centralized system fails
		\item Traffic volume: Bottleneck
		\item Distant centralized database: Longer delay for connections from far places from the system
	\end{itemize}
	\item Therefore uses \textbf{distributed} \& \textbf{hierarchical} database
	\begin{itemize}
		\item Root DNS server on the top
		\item Top-level domains (TLD) on the next level (\texttt{com}, \texttt{org}, \texttt{edu})
		\item Authoritative domains on the next level
	\end{itemize}
	\item Database Query Procedure (Simplified)
	\begin{enumerate}
		\item Client wants IP for \texttt{www.amazon.com}
		\item Client queries root DNS server to find \texttt{com} DNS server
		\item Client queries \texttt{com} DNS server to get \texttt{amazon.com} DNS server
		\item Client queries \texttt{amazon.com} DNS server to get the IP address for \texttt{www.amazon.com}
	\end{enumerate}
\end{itemize}

\subsubsection{DNS Hierarchy}
\begin{itemize}
	\item Root Name Servers
	\begin{itemize}
		\item Directly answers requests for records in the root zone
		\item Answers requests by returning a list of the authoritative name servers for the appropriate top-level domain
		\item 13 of them worldwide
	\end{itemize}
	\item Top Level Domains (TLD)
	\begin{itemize}
		\item Responsible for \texttt{com}, \texttt{org}, \texttt{net}, \texttt{edu} ...
		\item All top-level contry domains like \texttt{uk}, \texttt{fr}, \texttt{ca}, \texttt{kr} ...
	\end{itemize}
	\item Authoritative Servers
	\begin{itemize}
		\item An organization's own DNS server(s), prociding authoritative hostname to IP mappings for organization's named hosts
		\item Maintained by organization itself or service provider
	\end{itemize}
	\item Local DNS Servers
	\begin{itemize}
		\item Does not strictly belong to DNS hierarchy
		\item Each ISP has on also called ``default name server"
		\item When the client makes a DNS query, it is sent to its local DNS server
		\item The local DNS server caches the results of recent queries (name to address translation pairs)
		\item Can also act as a proxy, forwards query into hierarchy
	\end{itemize}
\end{itemize}

\subsubsection{DNS Name Resolution}
\begin{itemize}
	\item Situation: Host at \texttt{cis.poly.edu} wants IP for \texttt{gaia.cs.umass.edu}
	\item \textbf{Iterated Query}
	\begin{itemize}
		\item Contacted server replies with the name of server to contact
		\item \textit{I don't know this name, but ask this server}
		\item In the procedure below, we assume that no results are cached. If there is a cached result during the process, that result can be used. Then some steps can be skipped
	\end{itemize}
	\begin{enumerate}
		\item Client sends a request to local DNS server
		\item The local DNS server asks the root DNS server for the \texttt{edu} TLD DNS server
		\item The root DNS server replies
		\item The local DNS server asks the \texttt{edu} TLD DNS server for the authoritative DNS server
		\item The local DNS server asks the authoritative DNS server for the IP of \texttt{gaia.cs.umass.edu}
		\item The local DNS server replies to the client with the IP address
	\end{enumerate}
	\item \textbf{Recursive Query}
	\begin{itemize}
		\item Contacted server requests another server
		\item Puts the burden of name resolution on the contacted name server
		\item Not recommended due to heavy load at upper levels of hierarchy and security issues
		\item Systems on upper levels must wait for the result of each query - may result in a DoS attack
	\end{itemize}
	\begin{enumerate}
		\item Client sends a request to local DNS server
		\item The local DNS server asks the root DNS server
		\item The root DNS server asks the \texttt{edu} TLD DNS server
		\item The \texttt{edu} TLD DNS server asks the authoritative DNS server
		\item The authoritative DNS server replies with the IP address
		\item The \texttt{edu} TLD DNS server responds to the root DNS server with the IP address
		\item The root DNS server responds to the local DNS server with the IP address
		\item The local DNS server replies to the client with the IP address
	\end{enumerate}
\end{itemize}

\subsubsection{Attacking DNS}
\begin{itemize}
	\item DDoS Attacks
	\begin{itemize}
		\item Attacking root DNS servers with traffic - doesn't work well since local DNS servers cache IPs of TLD servers (doesn't connect to the root server)
		\item Attacking TLD servers can be more dangerous
	\end{itemize}
	\item Amplification Attacks
	\begin{itemize}
		\item Tricks the DNS server by putting the victim's IP inside the query
		\item The DNS server will send the result of the query to the victim's IP
	\end{itemize}
	\item Pharming Attacks
	\begin{itemize}
		\item Private data + Farming
		\item Domain hijacking, DNS poisoning
		\item The attacker breaks into the local DNS and modifies some mapping to the attacker's fake website
		\item The user may connect to that fake website and may enter private information
	\end{itemize}
\end{itemize}

\subsection{Peer-to-Peer Application}
\subsubsection{P2P Architecture}
\begin{itemize}
	\item No always-on server
	\item Arbitrary end systems communicate directly
	\item Peers are intermittently connected and change IP addresses
	\item P2P is scalable, compared to client-server model
\end{itemize}

\subsubsection{BitTorrent}
\begin{itemize}
	\item File is divided into 256 KB chunks
	\item Peers in torrent send/receive file chunks
	\item Torrent: group of peers exchanging chunks of a file
	\item Tracker: tracks peers participating in torrent
	\item A new client arrives, obtains list of peers from tracker, and begins exchanging file chunks with peers
	\item Chunk Receiving
	\begin{itemize}
		\item At any given time, different peers have different subsets of file chunks
		\item Periodically, client asks each peer for list of chunks that they have
		\item Client requests missing chunks from peers, \textit{rarest first}
	\end{itemize}
	\item Chunk Sending: \textit{tit-for-tat}
	\begin{itemize}
		\item To prevent free-riders
		\item The sender sends chunks to 4 other peers that are currently sending chunks to itself at highest rate
		\item Other peers will not receive chunks
		\item The top 4 peers are evaluated every 10 seconds
		\item For newly connected peers to receive chunks, the sender selects a random peer every 30 seconds, and starts sending chunks to it
		\item Now the newly connected peer may be included in the top 4 peers
		\item More upload results in better peers, resulting in faster download of data
	\end{itemize}
\end{itemize}

\subsection{Video Streaming and CDNs}
\subsubsection{Content Distribution Networks}
\begin{itemize}
	\item Video traffic is the major consumer of Internet bandwidth
	\item Challenge: \textbf{Scalability} - If we only use a single video server ...
	\begin{itemize}
		\item Single point of failure
		\item Point of network congestion
		\item Longer path to distant clients
		\item Multiple copies of same video sent over outgoing link
	\end{itemize}
	\item Solution: \textit{multiple copies of videos at multiple geographically distributed sites}
	\item CDN servers contain multiple copies of the same content, and lets the client stream the content data from the nearest/fastest CDN server
\end{itemize}




