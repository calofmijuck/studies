\section{컴퓨터 시스템의 소개}
\subsection{컴퓨터 하드웨어의 구성}
\begin{itemize}
	\item \textbf{하드웨어}: 데이터를 처리하는 물리적 기계장치
	\item \textbf{소프트웨어}: 작업을 지시하는 명령어로 작성한 프로그램
	\item \textbf{운영체제}: \textbf{하드웨어를 관리하는 소프트웨어}
\end{itemize}

\subsubsection{프로세서 (CPU)}
\begin{itemize}
	\item Central Processing Unit
	\item 컴퓨터 하드웨어에 부착된 모든 장치의 동작을 제어하고 명령을 실행한다
	\item 연산장치, 제어장치, 레지스터로 구성
\end{itemize}

\subsubsection{메모리}
\begin{itemize}
	\item Memory Hierarchy
	\item (빠른 속도/적은 용량) 레지스터 - 캐시 - 메인 메모리 - 보조 기억 장치 (느린 속도/큰 용량)
	\item 비용, 속도, 용량이 다른 메모리를 효과적으로 사용하여 시스템 성능 향상
	\item \textbf{레지스터}: 프로세서 내부, 가장 빠른 메모리
	\item \textbf{메인 메모리}
	\begin{itemize}
		\item 프로세서 외부에 있으며, DRAM 주로 사용
		\item 다수의 셀(cell)로 구성, 셀이 $k$ 비트이면 셀에 $2^k$ 종류의 값을 저장 가능
		\item 셀은 주소로 참조되며, $n$ 비트 주소라면 주소 범위는 $0 \sim 2^n-1$ (physical address)
		\item Virtual memory address 와 mapping 되어 사용
		\item \textbf{메모리 접근 시간}: 명령이 발생한 후 주소를 검색하여 읽기/쓰기를 시작할 때까지 걸린 시간
		\item \textbf{메모리 사이클 시간}: 두 번의 연속적인 메모리 동작 사이에 필요한 최소 지연 시간
	\end{itemize}
	\item \textbf{캐시 (cache)}
	\begin{itemize}
		\item 프로세서와 메인 메모리의 속도 차이를 보완하는 고속 버퍼
		\item 데이터를 block 단위로 가져와 프로세서에 워드 단위로 전달하여 속도를 높인다
		\item 프로세서가 참조하려는 정보가 캐시에 존재하면 cache hit, 존재하지 않으면 cache miss
		\item 캐시가 동작하는 이유는 \textbf{지역성(locality)} - 공간적(spatial), 시간적(temporal)
		\item Spatial locality: 참조한 주소와 인접한 주소의 내용을 다시 참조하는 특성
		\item Temporal locality: 한 번 참조한 주소를 곧 다시 참조하는 특성
	\end{itemize}
\end{itemize}

\subsubsection{시스템 버스}
\begin{itemize}
	\item 하드웨어를 물리적으로 연결하여 데이터를 주고받게 하는 통로
	\item 다양한 신호를 시스템 버스로 전달
	\item 데이터 버스, 주소 버스, 제어 버스
\end{itemize}

\subsubsection{주변 장치}
\begin{itemize}
	\item 나머지 하드웨어 구성 요소
	\item 입력 장치: 데이터를 외부에서 입력하는 장치
	\item 출력 장치: 처리한 데이터를 외부로 보내는 장치
\end{itemize}

\subsection{컴퓨터 시스템의 동작}
\begin{enumerate}
	\item 입력장치로 정보를 입력받아 메모리에 저장
	\item 메모리에 저장한 정보를 프로그램 제어에 따라 인출하여 연산장치에서 처리
	\item 처리한 정보를 출력 장치에 표시하거나 보조 기억장치에 저장
\end{enumerate}

\pagebreak